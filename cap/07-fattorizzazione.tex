\chapter{Sistemi di Fattorizzazione}
\begin{itemize}
	\item ogni funzione si può fattorizzare come composizione di una funzione iniettiva e una suriettiva;
	\item monomorfismi ed epimorfismi nella categoria degli insiemi si determinano a vicenda nel senso che segue:
	\Todo{lemma lemmino stammi vicino}
\end{itemize}
Questo motiva la nozione di \emph{ortogonalità} tra morfismi e di \emph{fattorizzazione} di ogni morfismo $f : X\to Y$ di una categoria $\ctC$ come una certa composizione $f = m\circ e : X \epi F \mono Y$.

Idealmente, una struttura del genere imposta su una categoria permette di `generarla' a partire dalle classi $\ctE,\ctM$; in una notazione suggestiva, $\hom(\ctC)=\ctM\circ \ctE$.
\section[Ortogonalità]{La relazione di ortogonalità}
Sulla classe dei morfismi di una generica categoria \(\ctC\) è possibile definire una relazione binaria detta \emph{ortogonalità}, nel modo che segue.
\begin{definition}
	Dati due morfismi \(f : X\to Y\) e \(g : A\to B\) diciamo che \(f\) è ortogonale a sinistra a \(g\), o che \(g\) è ortogonale a destra a \(f\), o che \(f\) e \(g\) sono \emph{ortogonali} se in ogni quadrato commutativo della forma
	\[
		\vcenter{\xymatrix{%
				X \ar[r]^u\ar[d]_f & A\ar[d]^g\\
				Y \ar[r]_v & B
			}}\]
	esiste un unico morfismo \(a : Y\to A\) tale che \(a\cmp f=u\) e \(g\cmp a=v\).
\end{definition}
La relazione di ortogonalità è un sottoinsieme \(\mathbin{\perp^\ctC} \subseteq \hom(\ctC)\) e il fatto che \((f,g)\in \mathbin{\perp^\ctC}\) si denota con il simbolo infisso \(f\perp^\ctC g\) o semplicemente come \(f\perp g\) quando il contesto permette di determinare quale sia la categoria \(\ctC\) di riferimento.

La relazione di ortogonalità è molto lontana dall'essere riflessiva, simmetrica o transitiva; \autoref{} sostanzia questa osservazione con delle maniere di costruire dei controesempi.
\begin{remark}\label{perche_ortogonale}
	Diverse intuizioni sulla nomenclatura `ortogonale'.
\end{remark}

\section[Proprietà ed esempi]{Proprietà fondamentali ed esempi}
\begin{itemize}
	\item estensione ai sottoinsiemi dell'ortogonalità
	\item Caratterizzazioni equivalenti della relazione di ortogonalità
	\item def di prefact, interdefinibilità delle due classi
	\item funtorialità della fattorizzazione
	\item stabilità per lim/colim
	\item proprietà di cancellazione
	\item classi larghe, sature, cellulari
	\item induzione di pre-FS su categorie di funtori, slice, coslice, (comma, cocomma?)
	\item esempi in algebra, topologia, geometria, logica, eccetera (epi e mono in set; verticali e cartesiani; etc)
	\item il poset dei sistemi di fattorizzazione
\end{itemize}
\section[Fattorizzazione]{Sistemi di fattorizzazione forti e deboli}
Ortogonalità non-unica
\begin{itemize}
	\item fattorizzabilità unica ed essenzialmente unica;
	\item fattorizzabilità non-unica
	\item esempi di sistemi con fattorizzabilità non-unica
	\item indurre sistemi di fattorizzazione (argomento degli oggetti piccini)
	\item
\end{itemize}
\section[Riflessività]{Sistemi di fattorizzazione riflessivi}
\section[Mono ed epimorfismi]{Classi di mono ed epimorfismi}
\section[Iniettivi e proiettivi]{Oggetti iniettivi e proiettivi}
\section[Fattorizzazione e algebre]{Sistemi di fattorizzazione come algebre}
\subsubsection*{Esercizi}
\begin{enumerate}
	\item
	\item
	\item
	\item
	\item
\end{enumerate}

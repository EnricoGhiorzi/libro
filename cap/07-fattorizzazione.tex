\chapter{Sistemi di Fattorizzazione}
\section[Ortogonalità]{La relazione di ortogonalità}
Sulla classe dei morfismi di una generica categoria $\ctC$ è possibile definire una relazione binaria detta \emph{ortogonalità}, nel modo che segue.
\begin{definition}
    Dati due morfismi $f : X\to Y$ e $g : A\to B$ diciamo che $f$ è ""ortogonale a sinistra"" a $g$, o che $g$ è ""ortogonale a destra"" a $f$, se in ogni quadrato della forma
    \[
        \vcenter{\xymatrix{A \ar[r] & B}}
    \]
esiste un unico morfismo $a : Y\to A$ tale che $a\cmp f=u$ e $g\cmp a=v$.
\end{definition}
La relazione di ortogonalità è un sottoinsieme $\mathbin{\perp^\ctC} \subseteq \hom(\ctC)$ e il fatto che $(f,g)\in \mathbin{\perp^\ctC}$ si denota con il simbolo infisso $f\perp^\ctC g$ o semplicemente come $f\perp g$ quando il contesto permette di determinare quale sia la categoryia \(\ctC\) di riferimento.

La relazione di ortogonalità è molto lontana dall'essere riflessiva, simmetrica o transitiva; \autoref{} sostanzia questa osservazione con delle maniere di costruire dei controesempi.
\begin{remark}
Diverse intuizioni sulla nomenclatura `ortogonale'.
\end{remark}

\section[Proprietà ed esempi]{Proprietà fondamentali ed esempi}
\section[Fattorizzazione]{Sistemi di fattorizzazione ortogonali e deboli}
\section[Riflessività]{Sistemi di fattorizzazione riflessivi}
\section[Mono ed epimorfismi]{Classi di mono ed epimorfismi}
\section[Iniettivi e proiettivi]{Oggetti iniettivi e proiettivi}
\section[Fattorizzazione e algebre]{Sistemi di fattorizzazione come algebre}
\subsubsection*{Esercizi}
\begin{enumerate}
    \item
    \item
    \item
    \item
    \item
\end{enumerate}

\chapter{Categorie, funtori, trasformazioni naturali}
\subsubsection*{Esercizi}
\begin{enumerate}
    \item 
    \item 
    \item 
    \item 
    \item 
\end{enumerate}

\section{Monomorfismi ed epimorfismi}

Siccome i monoidi corrispondono precisamente alle categorie con un solo oggetto, come visto nell'esempio~\ref{???},
la definizione di elemento di un monoide cancellabile a sinistra o a destra si può applicare direttamente ai morfismi delle categorie da loro indotte.
Ricordiamo che un elemento \(x\) di un monoide \((M, \cdot, 1)\) si dice cancellabile a sinistra (destra) se,
per ogni coppia di elementi \(y_0\) e \(y_1\) in \(M\) tali che \(x \cdot y_0 = x \cdot y_1\) (\(y_0 \cdot x = y_1 \cdot x\)),
si ha che \(y_0 = y_1\).
Se però vogliamo estendere tale definizione ad una categoria generica,
la cui composizione è dunque un'operazione parziale,
è necessario considerare solo le composizioni ben definite
(ovvero tra frecce consecutive).
Si arriva così alle seguenti definizioni.

\begin{definition}[Monomorfismo]
	Un morfismo \(m \colon B \to X\) in una categoria \(\ctC\) è un \emph{monomorfismo} (o \emph{mono}) se,
	per ogni coppia di frecce parallele \(f, g \colon A \to B\) in \(\ctC\) tali che \(m \cmp f = m \cmp g\), si ha che \(f = g\).
	%Equivalentemente, \(m\) è un monomorfismo se è cancellabile a sinistra, ovvero \(m \cmp f = m \cmp g\) implica \(f = g\) ogni qual volta la scrittura ha senso (ovvero quando \(f\) e \(g\) sono frecce parallele e pre-componibili con \(m\)).
\end{definition}

\begin{definition}[Epimorfismo]
	Un morfismo \(e \colon X \to A\) in una categoria \(\ctC\) è un \emph{epimorfismo} (o \emph{epi}) se,
	per ogni coppia di frecce parallele \(f, g \colon A \to B\) in \(\ctC\) tali che \(f \cmp e = g \cmp e\), si ha che \(f = g\).
	%Equivalentemente, \(e\) è un epimorfismo se è cancellabile a destra, ovvero \(f \cmp e = g \cmp e\) implica \(f = g\) ogni qual volta la scrittura ha senso (ovvero quando \(f\) e \(g\) sono frecce parallele e post-componibili con \(e\)).
\end{definition}

Siccome le definizioni categoriali di mono ed epi generalizzano, rispettivamente, quelle, algebriche, di elementi cancellabili a sinistra e a destra,
otteniamo immediatamente il seguente esempio.

\begin{example}
	Nella categoria indotta da un monoide, i mono sono gli elementi cancellabili a sinistra,
	e gli epi sono gli elementi cancellabili a destra.
\end{example}

Vediamo ora invece esempi in categorie non indotte da monoidi.
Il primo caso, alquanto degenere, è dato dai preordini. 

\begin{example}
	Nella categoria indotta da un preordine, come visto nell'esempio~\ref{???},
	tutti i morfismi sono sia epi che mono.
	Infatti, la definizione di epi e mono è sempre soddisfatta banalmente,
	dato che, in un preordine, due frecce parallele sono necessariamente uguali.
\end{example}

Gli esempi precedenti mostrano come
un morfismo possa essere sia epi che mono, e comunque non essere un isomorfismo.
Il prossimo esempio, rispetto ai precedenti, è più complesso.

% TODO: Lo svolgimento dell'esempio dovrebbe invece essere un esercizio svolto? Come lo formattiamo?
\begin{example}
	In \(\ctSet\), la categoria degli insiemi, i monomorfismi sono precisamente le funzioni iniettive
	e gli epimorfismi sono precisamente le funzioni surgettive.
	
	Consideriamo il caso delle funzioni surgettive.
	Se \(e \colon X \to A\) è una funzione surgettiva e \(f, g \colon A \to B\) sono funzioni tali che \(f \cmp e = g \cmp e\),
	allora per ogni elemento \(a \in A\) esiste un \(x \in X\) tale che \(e(x) = a\), per la surgettività di \(e\).
	Osserviamo che \((f \cmp e)(x) = (g \cmp e)(x)\), ovvero \(f(e(x)) = g(e(x))\),
	ma allora \(f(a) = g(a)\).
	Per la genericità di \(a\) concludiamo che \(f = g\), e dunque \(e\) è epi.

	Nel caso opposto, in cui \(e \colon X \to A\) è epi, sia \(a \in A\).
	Assumiamo per assurdo che non esista \(x \in X\) tale che \(f(x) = a\).
	Allora siano \(k_0 \colon A \to \{0, 1\}\) la funzione costante in \(0\),
	e \(\delta_a \colon A \to \{0, 1\}\) la funzione caratteristica di \(a\) in \(A\).
	Siccome \(a\) non appartiene all'immagine di \(e\), abbiamo che \(k_0 \cmp e = \delta_a \cmp e\)
	in quanto entrambe le composizioni assumono costantemente il valore \(0\) su tutto il dominio \(X\).
	Dunque, \(k_0 = \delta_a\), il che è assurdo.
	Dalla negazione dell'ipotesi per assurdo, concludiamo che \(e\) è surgettiva.
\end{example}

Si noti che la dimostrazione che le funzioni surgettive sono epi sarebbe immediata
se usassimo il fatto che ogni funzione surgettiva ha un'inversa destra.
L'esistenza dell'inversa destra, però, è una conseguenza dell'assioma di scelta,
che invece non viene impiegato nella dimostrazione di cui sopra.

\begin{exercise}
	Dimostrare che una funzione è iniettiva se e solo se è un mono in \(\ctSet\),
	ma senza utilizzare il fatto che le funzioni iniettive hanno inversa sinistra.
\end{exercise}

Per ragioni che saranno evidenti nella prossima sezione,
abbiamo evidenziato come l'equivalenza tra funzioni surgettive (iniettive)
e epi (mono) in \(\ctSet\) sia indipendente dall'esistenza di un'inversa destra (sinistra).

\begin{remark}
\label{rmk:mono-epi-duality}
Le nozioni di monomorfismo ed epimorfismo sono duali:
un morfismo in una categoria \(\ctC\) è mono (epi) se e solo se è epi (mono) in \(\ctC^{\op}\) (vedi \ref{???}).
\end{remark}

Dimostriamo ora alcuni risultati che rendono più agevole lavorare con mono ed epi,
sfruttando la dualità a nostro vantaggio.

\begin{proposition}
	Siano \(f \colon A \to B\) e \(g \colon B \to C\) due frecce consecutive in una qualche categoria.
	Allora:
	\begin{enumerate}
		\item Se \(f\) e \(g\) sono entrambe mono (epi), allora anche \(g \cmp f\) è mono (epi).
		\item Se \(g \cmp f\) è mono, allora anche \(f\) è mono.
		\item Se \(g \cmp f\) è epi, allora anche \(g\) è epi.
	\end{enumerate}
\end{proposition}
\begin{proof}
	Grazie all'osservazione~\ref{rmk:mono-epi-duality},
	ci basta dimostrare il risultato per i mono,
	e il risultato per gli epi segue per dualità.
	\begin{enumerate}
		\item Siano \(f\) e \(g\) mono.
		Per dimostrare che \(g \cmp f\) è mono,
		consideriamo una coppia di frecce parallele \(h\) e \(k\) tali che \((g \cmp f) \cmp h = (g \cmp f) \cmp k\).
		Per associatività della composizione, \(g \cmp (f \cmp h) = g \cmp (f \cmp k)\).
		Allora, siccome \(g\) per ipotesi è mono, si ha che \(f \cmp h = f \cmp k\).
		Inoltre, siccome \(f\) per ipotesi è mono, si ha che \(h = k\).
		Quindi, per la generalità di \(h\) e \(k\), si ha che anche \(g \cmp f\) è mono.
		\item Assumiamo che \(g \cmp f\) sia mono.
		Per dimostrare che \(f\) è mono,
		consideriamo una coppia di frecce parallele \(h\) e \(k\) tali che \(f \cmp h = f \cmp h\).
		Allora, post-componendo con \(g\) su ambo i lati e sfruttando la associatività della composizione,
		otteniamo che \((g \cmp f) \cmp h = (g \cmp f) \cmp k\).
		Siccome per ipotesi \(g \cmp f\) è mono, si ha che \(h = k\).
		Quindi, per la generalità di \(h\) e \(k\), si ha che anche \(f\) è mono. \qedhere
	\end{enumerate}
\end{proof}



\section{Sezioni e retrazioni}

\begin{definition}[Sezione]
	Una \emph{sezione} di un morfismo \(f \colon A \to B\) in una categoria \(\ctC\) è un'inversa destra di \(f\), ovvero un morfismo \(s \colon B \to A\) tale che \(f \cmp s = \id_{B}\).
\end{definition}

\begin{definition}[Retrazione]
	Una \emph{retrazione} di un morfismo \(f \colon A \to B\) in una categoria \(\ctC\) è un'inversa sinistra di \(f\), ovvero un morfismo \(r \colon B \to A\) tale che \(r \cmp f = \id_{A}\).
\end{definition}

Le nozioni di sezione e retrazione sono intrecciate tra di loro:
un morfismo che ha una sezione ne è una retrazione,
e viceversa.

\begin{example}
	In \(\ctSet\), una funzione ha una retrazione se e solo se è iniettiva.
	Inoltre, le funzioni che hanno una sezione sono surgettive,
	mentre l'implicazione inversa vale soltanto se assumiamo l'Assioma di Scelta.
	% TODO: questo è il teorema di Cantor? Riferimento necessario?

	% Questo esempio, inoltre, dimostra che sezioni e retrazioni non sono uniche:
	% una funzione surgettiva può avere più inverse destre (sezioni).
\end{example}
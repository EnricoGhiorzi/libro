\chapter{Categorie, funtori, trasformazioni naturali}

\section{Categorie}\label{categorie}

Prima di arrivare alla definizione formale di categorie, può essere di aiuto avere in mente un esempio concreto. Prendiamo per esempio gli spazi vettoriali e le mappe lineari tra di loro. 
Molti di voi sapranno che le matrici invertibili su $\ictR^n$ con il loro prodotto formano un gruppo. Più un generale, se tralasciamo la condizione di invertibilità, abbiamo un \emph{monoide}:

\begin{definition}
 Un \emph{monoide} consiste in un insieme $M$ con:
 \begin{itemize}
  \item Un elemento particolare $e\in M$, chiamato \emph{elemento neutro}, \emph{unità} o \emph{identità};
  \item Un'operazione binaria $M\times M\to M$, chiamata \emph{prodotto} o \emph{composizione}, e indicata con $(m,n),\mapsto m\cdot n$;
  che soddisfano le seguenti proprietà:
  \item \emph{Unitalità}: per ogni $m\in M$, i prodotti $m \cdot e$ ed $e\cdot m$ sono uguali a $m$;
  \item \emph{Associatività}: per ogni $m,n,p\in M$, i prodotti $(m\cdot n)\cdot p$ ed $m\cdot (n\cdot p)$ sono uguali. 
 \end{itemize}
\end{definition}

Se ora consideriamo gli spazi vettoriali $\ictR^n$ per diversi $n$, non tutte le mappe lineari si possono comporre. In particolare, una mappa $f:\ictR^n\to \ictR^m$ si può comporre con una mappa $g:\ictR^p\to\ictR^q$ solo se $m=p$. In altre parole, $g$ si può comporre con $f$ solo se il codominio di $f$ è uguale al dominio di $g$.
Inoltre, non c'e solo una matrice identità, ma ce n'è una per ogni $n$. 
La struttura algebrica che otteniamo è una \emph{categoria}, che definiamo a breve.

Nella definizione di categoria, in generale, è problematico usare la parola ``insieme'', è meglio parlare di \emph{classe}. Informalmente, una classe è come un insieme, ma potenzialmente ``più grande'', cioè può contenere tutti gli insiemi (ma non tutte le classi). 
Nella definizione che segue, a prima vista si può leggere la parola ``classe'' come ``insieme''. Vedremo la questione più in dettaglio nell'Appendice \ref{fondamenti}. 

\begin{definition}
 Una \emph{categoria} $\ctC$ consiste dei seguenti dati.
 \begin{itemize}
  \item Una classe $\ctC_0$ i cui elementi chiamiamo \emph{oggetti}, di solito indicati con lettere latine maiuscole: $A$, $B$, $X$, $Y$,\dots
  \item Una classe $\ctC_1$ i cui elementi chiamiamo \emph{morfismi}, di solito indicati con lettere latine minuscole: $f$, $g$, $h$,\dots
  
  Queste classi sono fornite delle seguenti strutture. 
  
  \item Ad ogni morfismo corrispondono due oggetti $\dom{f}$, $\cod{f}$ chiamati \emph{dominio} e \emph{codominio}. Se $f$ ha dominio $X$ e codominio $Y$, scriviamo $f:X\to Y$, o in forma diagrammatica, $X \xrightarrow{f} Y$.
  \item Ogni oggetto $X$ ha un particolare morfismo $\id_X:X\to X$ chiamato \emph{identità}.
  \item Dati tre oggetti $X,Y,Z$ e morfismi $f:X\to Y$ e $g:Y\to Z$ (si noti che $\cod{f}=\dom{g}$) esiste un particolare morfismo $g\circ f:X\to Z$ chiamato \emph{composizione di $f$ e $g$}. Graficamente:
  $$
  \begin{tikzcd}
   X \ar{r}{f} \ar[bend right=20]{rr}[swap]{g\circ f} & Y \ar{r}{g} & Z .
  \end{tikzcd}
  $$
  (Si noti che $\dom{g\circ f}=\dom{f}$ e $\cod{g\circ f}=\cod{g}$.)
  
  Queste strutture devono soddisfare le seguenti proprietà.
  
  \item \emph{Unitalità}: per ogni morfismo $f:X\to Y$, le composizioni $f\circ\id_X$ e $\id_Y\circ f$ sono uguali ad $f$.
  \item \emph{Associatività}: dati oggetti $X,Y,Z,A$ e morfismi $f:X\to Y$, $g:Y\to Z$ e $h:Z\to A$, le composizioni $h\circ (g\circ f)$ e $(h\circ g)\circ f$ sono uguali.
 \end{itemize}
\end{definition}

Dati due oggetti $X$ e $Y$, indichiamo con $\Hom{\ctC}(X,Y)$ la classe di morfismi da $X$ a $Y$. Altri autori usano altre notazioni, come $\mathrm{Hom}(X,Y)$ o $\mathrm{Hom}_\ctC(X,Y)$.


Quando le classi $\ctC_0$ e $\ctC_1$ sono insiemi, chiamamo la categoria $\ctC$ \emph{piccola}. Più in generale, $\ctC$ si dice \emph{localmente piccola} se dati ogni due oggetti $X$ e $Y$, la classe $\Hom{\ctC}(X,Y)$ di morfismi da $X$ a $Y$ è un insieme. 

\begin{examples} 
 Vediamo alcuni esempi classici di categorie. 
 \begin{itemize}
  \item Come accennato sopra, gli spazi vettoriali su $\ictR$ e le mappe lineari formano una categoria, che chiamiamo $\ctVect_\ictR$. Più in generale possiamo definire la categoria $\ctVect_\ictF$, per un campo $\ictF$, in maniera analoga.
  \item Gli insiemi e le funzioni formano una categoria, che chiamiamo $\ctSet$.
  \item I gruppi e gli omomorfismi di gruppo formano la categoria $\ctGrp$.
  \item I gruppi abeliani e gli omomorfismi di gruppo formano la sottocategoria $\ctAb\subseteq\ctGrp$. (Definiremo le sottocategorie a breve.)
  \item Gli spazi topologici e le funzioni continue formano la categoria $\ctTop$. 
 \end{itemize}
Spesso un ambito della matematica si occupa di una o più categorie in particolare. Per esempio l'algebra lineare si occupa principalmente delle categorie $\ctVect_\ictF$.
\end{examples}

Le categorie negli esempi sopra si possono vedere come categorie dove gli oggetti sono insiemi con strutture ulteriori (per esempio, strutture di gruppo), e i morfismi sono funzioni che preservano queste strutture. Non tutte le categorie sono di questo tipo, come vediamo nei prossimi esempi. 

\begin{example}
 Una relazione di equivalenza si può vedere come una particolare categoria. Consideriamo un insieme $X$ con una relazione di equivalenza indicata con il simbolo $\sim$.
 Definiamo la seguente categoria:
 \begin{itemize}
  \item Gli oggetti sono gli elementi di $x$;
  \item Esiste un unico morfismo da $x$ a $y$ (e uno da $y$ a $x$) se e solo se $x\sim y$. 
 \end{itemize}
 Ad ogni oggetto $x$, l'identità è l'unico morfismo $x\to x$ dato dal fatto che la relazione è riflessiva ($x\sim x$). 
 La composizione è data dalla transitività: se abbiamo morfismi $x\to y$ e $y\to z$ significa, in particolare, che $x\sim y$ e $y\sim z$. Per transitività, $x\sim z$, e quindi c'è un unico morfismo $x\to z$, che possiamo prendere come composizione. 
 Gli assiomi di unitalità e associatività sono automaticamente soddisfatti: per esempio, dati $w\sim x\sim y\sim z$, per transitività abbiamo che $w\sim z$, e questo dà un \emph{unico} morfismo $w\to z$, non importa se lo otteniamo componendo prima $w\sim x$ e $x\sim y$ o prima $x\sim y$ e $y\sim z$ -- il risultato è lo stesso per unicità. 
\end{example}

Nell'esempio sopra vediamo che \emph{gli oggetti non sono insiemi con strutture ulteriori}. 
Notiamo anche che la proprietà di \emph{simmetria} delle relazioni di equivalenza non serve per avere una categoria. Consideriamo i seguenti esempi.

\begin{examples}
 Una relazione d'ordine $(X,\le)$ si può vedere come una categoria, ancora una volta con un unico morfismo $x\to y$ se e solo se $x\le y$. Come nel caso delle relazioni di equivalenza, le identità sono date dalla proprietà riflessiva e la composizione dalla proprietà transitiva. 
 
 Più in generale, un \emph{preordine} è una relazione riflessiva e transitiva, ma non necessariamente simmetrica o antisimmetrica. Ogni preordine si può vedere come una categoria.
\end{examples}


\begin{example}
 Un monoide si può vedere come una categoria con un solo oggetto -- più precisamente, una categoria \emph{localmente piccola} con un solo oggetto. 
 Vediamo come. Dato un monoide $M$, definiamo la seguente categoria, che indichiamo con $\ctB M$.\footnote{A volte la categoria definita in questo modo si indica con $\ctB M$, a volte semplicemente con $M$.}
 \begin{itemize}
  \item La categoria $\ctB M$ ha un unico oggetto, che indichiamo con un punto, $\bullet$;
  \item La categoria $\ctB M$ ha un morfismo $m:\bullet\to\bullet$ per ogni elemento $m\in M$;
  \item L'identità dell'unico oggetto $m$ e il morfismo definito dall'elemento neutro $e\in M$;
  \item La composizione di morfismi è data dal prodotto in $M$.
 \end{itemize}
 In un monoide, a differenza di una categoria generica, possiamo sempre comporre due morfismi $m$ e $n$: questo è garantito dal fatto che siccome c'è un solo oggetto, il dominio e il codominio di $m$ ed $n$ sono necessariamente uguali. 
 
 Viceversa, data una categoria localmente piccola con un solo oggetto (indichiamolo ancora con questo simbolo, $\bullet$), l'insieme di morfismi $\bullet\to\bullet$ ha una struttura di monoide con l'elemento neutro dato dall'identità, e il prodotto dato dalla composizione.
 
 L'unitalità e l'associatività del monoide e della categoria si corrispondono.
\end{example}

In particolare, ogni gruppo si può vedere come una categoria con un solo oggetto. 
Dato un oggetto $X$ di una categoria, un morfismo da $X$ a $X$ si chiama anche \emph{endomorfismo}. In un monoide, tutti i morfismi sono endomorfismi.

\begin{remark}
 Abbiamo visto che i gruppi sono gli oggetti della categoria $\ctGrp$, ma anche che ogni gruppo si può a sua volta vedere come una categoria (con un solo oggetto). Entrambe le prospettive sono valide, e ce ne sono altre ancora. 
 Allo stesso modo, gli ordini parziali sono categorie, ma si possono anche vedere come oggetti della categoria $\ctPos$, dove i morfismi sono le funzioni monotone. 
 Spesso, in teoria delle categorie, la stessa struttura matematica può apparire in modi diversi in contesti diversi, e questa varietà di prospettive è uno dei motivi per cui le categorie sono così versatili. 
\end{remark}

Concludiamo questa sezione con alcuni esempi estremi di categorie, utili per controesempi e casi limite.

\begin{examples}
 Consideriamo un insieme $X$. Possiamo considerarlo una categoria in due modi:
 \begin{itemize}
  \item La \emph{categoria discreta su $X$} ha come oggetti gli elementi di $X$, e gli unici morfismi sono le identità.
  \item La \emph{categoria codiscreta su $X$} ha come oggetti gli elementi di $X$, e un unico morfismo tra ogni due elementi.
 \end{itemize}
 La categoria discreta, equivalentemente, è la relazione di equivalenza discreta (la relazione di uguaglianza).
 La categoria codiscreta si può vedere come un'unica grande classe di equivalenza su $X$.
\end{examples}

\begin{example}
 La \emph{categoria terminale} o \emph{unitaria}, che indichiamo con $\cate{1}$, ha un solo oggetto e un solo morfismo (identità).
\end{example}

\begin{example}
 La \emph{categoria vuota} non ha oggetti né morfismi.
\end{example}

Tutte le categorie descritte finora sono localmente piccole. 

\subsubsection*{Esercizi}
\begin{enumerate}
    \item Un \emph{grafo diretto} è un insieme $V$, i cui elementi chiamiamo \emph{vertici}, e una relazione $E\subseteq X\times X$, i cui elementi chiamiamo \emph{spigoli}. Diciamo che $x$ è \emph{adiacente} a $y$ se esiste uno spigolo $(x,y)\in E$. Un \emph{omomorfismo di grafi diretti} $f:(V,E)\to (V',E')$ è una funzione $f:V\to V'$ tale che se $x$ è adiacente a $y$, allora $f(x)$ è adiacente a $f(y)$. Dimostra che i grafi diretti e i loro omomorfismi formano una categoria. 
    \item Un \emph{multigrafo diretto} è come un grafo diretto, ma può avere diversi spigoli tra gli stessi due vertici. Tecnicamente consiste di un insieme $V$ (i \emph{vertici}), e un insieme $E$ (gli \emph{spigoli}), con due funzioni $d_-,d_+:E\to V$, che associano un ``dominio'' e un ``codominio'' ad ogni spigolo. Un \emph{omomorfismo di multigrafi diretti} $(V,E)\to (V',E')$ consiste di una funzione $f_0:V\to V'$ tra i vertici, e una funzione $f_1:E\to E'$ tra gli spigoli, compatibili con le mappe $d_-,d_+$ nel seguente modo:
    $$
    d_-(f_1(e)) = f_0(d_-(e)) ,\qquad d_+(f_1(e)) = f_0(d_+(e)) .
    $$
    Dimostra che $f_1:E\to E'$ associa a uno spigolo tra $x$ e $y$ uno spigolo tra $f_0(x)$ e $f_0(y)$. 
    \item Dimostra che i multigrafi diretti e i loro omomorfismi formano una categoria. 
    \item Una categoria piccola si può vedere come un multigrafo con identità e composizione: riscrivi la definizione di categoria (piccola) in termini di multigrafi.  In che modo questa costruzione generalizza la definizione di monoide (come insieme con identità e composizione)?
    \item Un grafo diretto si può vedere come un multigrafo dove tra due vertici c'è al massimo uno spigolo. Dimostra che la costruzione al punto 3, se applicata ad un grafo diretto, dà un preordine. 
\end{enumerate}


\section{Isomorfismi}\label{isomorfismi}

\begin{definition}
 Dati due oggetti $X$ e $Y$ in una categoria $\ctC$, un \emph{isomorfismo} tra $X$ e $Y$ è una coppia di morfismi $f:X\to Y$ e $g:Y\to X$ tali che $g\circ f=\id_X$ e $f\circ g=\id_Y$. 
 $f$ è chiamato l'\emph{inverso} di $g$, e a volte si indica con $g^{-1}$, e viceversa.
 
 Se esiste un isomorfismo tra $X$ e $Y$ diciamo che i due oggetti sono \emph{isomorfi}.
\end{definition}

Può essere utile un'interpretazione grafica: intuitivamente, un isomorfismo è una situazione dove l'applicazione di $f$ si può invertire o annullare tramite $g$, e viceversa, e si può sempre passare da $X$ a $Y$ senza perdere informazioni. 
$$
\begin{tikzcd}
 X \ar[shift left]{r}{f} & Y \ar[shift left]{l}{g}
\end{tikzcd}
$$

Normalmente si indica la coppia $(f,g)$ solo con $f$ o con $g$, lasciando l'inverso implicito (l'inverso, se esiste, è unico, si vedano gli esercizi).

\begin{examples}
 La definizione di isomorfismo comprende molte delle nozioni di ``struttura equivalente'' usate in matematica:
 \begin{itemize}
  \item In $\ctSet$, gli isomorfismi sono le biiezioni, o corrispondenze biunivoche. In particolare, due insiemi sono isomorfi se e solo se hanno la stessa cardinalità.
  \item In $\ctVect_\ictR$, gli isomorfismi sono le biiezioni lineari. In particulare, in dimensione finita, due spazi vettoriali sono isomorfi se e solo se hanno la stessa dimensione.
  \item In $\ctTop$, gli isomorfismi sono gli omeomorfismi: funzioni continue con un'inversa continua. 
  \item In $\ctGrp$, gli isomorfismi sono gli isomorfismi di gruppi. 
 \end{itemize}
\end{examples}

\begin{warning}
 Spesso una categoria viene chiamata con il nome dei suoi oggetti ($\ctVect$, $\ctGrp$, eccetera), ma è importante sapere \emph{quali sono i morfismi} per sapere che struttura in particolare si sta studiando. Questo è particolarmente importante nel caso degli isomorfismi: per esempio, $\ictR$ e $\ictR^2$, come insiemi, sono isomorfi, ma non come spazi vettoriali. Per cui, nella categoria $\ctVect$, gli oggetti $\ictR$ e $\ictR^2$ sono isomorfi, ma non sono isomorfi se li intendiamo come oggetti della categoria $\ctSet$. 
\end{warning}

Si noti che le condizioni $g\circ f = \id_X$ e $f\circ g=\id_Y$ servono entrambe, perché possono valere o non valere indipendentemente l'una dall'altra. 
\begin{example}
 Nella categoria $\ctVect_\ictR$, si consideri la mappa $f:\ictR\to\ictR^2$ data dall'inclusione dell'asse $X$, $x\mapsto(x,0)$. Data la proiezione sull'asse $X$ data da $g:\ictR^2\to\ictR$, $(x,y)\mapsto x$, si ha che $g\circ f=\id_\ictR$. Però non è vero che $f\circ g = \id_{\ictR^2}$. (E infatti i due oggetti non sono isomorfi come spazi vettoriali.)
\end{example}

\begin{definition}
 Una categoria in cui tutti i morfismi sono isomorfismi si chiama \emph{grouppoide}.\footnote{In inglese: \emph{groupoid}.} 
\end{definition}

\begin{example}
 Abbiamo visto che un preordine (una relazione riflessiva e transitiva) si può vedere come una categoria. Questa categoria è un gruppoide se e solo se la relazione è anche simmetrica (cioè, è un'equivalenza). Se $x\sim y$, abbiamo un unico morfismo $x\to y$. Questo morfismo è invertibile se e solo se esiste un morfismo $y\to x$, cioè, se anche $y\sim x$. (Si noti che non ci sono altre condizioni da soddisfare per avere un inverso: per esempio, il morfismo composto $x\to y\to x$ è necessariamente uguale a $\id_x$ per unicità.)
\end{example}

\begin{example}
 Abbiamo visto che ogni monoide $M$ si può considerare una categoria con un solo oggetto. Questa categoria è un gruppoide se e solo se $M$ è un gruppo. Infatti, un morfismo $m$ ammette un inverso nel senso della teoria delle categorie se e solo se ammette un inverso nel senso della teoria dei gruppi: $g\cdot g^{-1}=g^{-1}\cdot g = e$. 
\end{example}

Un \emph{automorfismo} è un endomorfismo (un morfismo da un oggetto a sé stesso) invertibile. In un gruppo, tutti i morfismi sono automorfismi.

\begin{example}
 La categoria i cui oggetti sono gli insiemi e i cui morfismi sono biiezioni è un gruppoide.
\end{example}

L'esempio precedente si può vedere come una sottocategoria di $\ctSet$.
Più in generale, data una categoria, possiamo sempre prendere la sottocategoria che contiene solo gli isomorfismi.

\begin{definition} 
 Il \emph{cuore}\footnote{In inglese: \emph{core}} di una categoria $\ctC$ è il gruppoide ottenuto prendendo come oggetti gli oggetti di $\ctC$, e come morfismi gli isomorfismi di $\ctC$. 
\end{definition}

Si vedano gli esercizi sul perché in questo modo si ottenga una categoria (con identità e composizione). 


\subsubsection*{Esercizi}
\begin{enumerate}
    \item Dato un morfismo $f:X\to Y$ in una categoria $\ctC$, dimostra che se un'inverso per $f$ esiste, allora è unico.
    \item Dati due spazi metrici $X$ e $Y$, ricorda che una funzione $f:X\to Y$ si dice \emph{1-Lipschitz} se soddisfa la disuguaglianza
    $$
     d\big( f(x), f(x') \big) \le d(x,x')
    $$
    per ogni $x,x'\in X$. Dimostra che spazi metrici e le funzioni 1-Lipschitz formano una categoria, chiamata $\cate{Met}_\cate{Lip}$, e che gli isomorfismi di $\cate{Met}_\cate{Lip}$ sono le isometrie. Fornisci un esempio di due spazi metrici isomorfi in $\ctTop$, ma non in $\cate{Met}_\cate{Lip}$.
    \item Costruisci un esempio in $\ctTop$ di oggetti $X$ e $Y$ non isomorfi, e morfismi $f:X\to Y$ e $g:Y\to X$ tali che $g\circ f=\id_X$ (ma $f\circ g \ne \id_Y$). 
    \item Dimostra che in ogni categoria, le identità sono isomorfismi, e che la composizione di isomorfismi è un isomorfismo. 
    \item In una categoria localmente piccola $\ctC$, dati due oggetti isomorfi $X$ e $Y$, dimostra che per ogni oggetto $A$, gli insiemi $\Hom{C}(A,X)$ e $\Hom{C}(A,Y)$ sono in corrispondenza biunivoca.
\end{enumerate}


\section{Funtori e diagrammi}

\subsubsection*{Esercizi}
\begin{enumerate}
    \item 
    \item 
    \item 
    \item 
    \item 
\end{enumerate}


\section{Operazioni tra categorie}

% categoria opposta, funtori controvarianti, prefasci
% categoria prodotto, bifuntori
% hom
% categorie slice
% spazi puntati
% scheletri

\subsubsection*{Esercizi}
\begin{enumerate}
    \item 
    \item 
    \item 
    \item 
    \item 
\end{enumerate}


\section{Trasformazioni naturali}

\subsubsection*{Esercizi}
\begin{enumerate}
    \item 
    \item 
    \item 
    \item 
    \item 
\end{enumerate}

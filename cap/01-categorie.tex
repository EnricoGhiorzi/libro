\chapter{Categorie, funtori, trasformazioni naturali}
\subsubsection*{Esercizi}
\begin{enumerate}
    \item 
    \item 
    \item 
    \item 
    \item 
\end{enumerate}

\section{Monomorfismi ed epimorfismi}

Un monoide, non essendo necessariamente commutativo, può avere sia elementi cancellabili a sinistra,
sia elementi cancellabili a destra.
Un elemento \(x\) di un monoide \((M, \cdot, 1)\) è cancellabile a sinistra se,
per ogni coppia di elementi \(y_0\) e \(y_1\) in \(M\) tali che \(x \cdot y_0 = x \cdot y_1\),
si ha che \(y_0 = y_1\),
e gli elementi cancellabili a destra sono definiti analogamente.
Siccome i monoidi corrispondono precisamente alle categorie con un solo oggetto (\ref{???}),
la definizione di elementi cancellabili a destra o sinistra si può applicare direttamente ai morfismi di tali categorie.
Estendiamo ora questa definizione ad una categoria generica,
dove dunque possono esserci più oggetti e, di conseguenza, la composizione è un'operazione parziale.

\begin{definition}[Monomorfismo]
	Un morfismo \(m \colon B \to X\) in una categoria \(\ctC\) è un \emph{monomorfismo} (o \emph{mono}) se,
	per ogni coppia di frecce parallele \(f, g \colon A \to B\) in \(\ctC\) tali che \(m \cmp f = m \cmp g\), si ha che \(f = g\).
	%Equivalentemente, \(m\) è un monomorfismo se è cancellabile a sinistra, ovvero \(m \cmp f = m \cmp g\) implica \(f = g\) ogni qual volta la scrittura ha senso (ovvero quando \(f\) e \(g\) sono frecce parallele e pre-componibili con \(m\)).
\end{definition}

\begin{definition}[Epimorfismo]
	Un morfismo \(e \colon X \to A\) in una categoria \(\ctC\) è un \emph{epimorfismo} (o \emph{epi}),
	per ogni coppia di frecce parallele \(f, g \colon A \to B\) in \(\ctC\) tali che \(f \cmp e = g \cmp e\), si ha che \(f = g\).
	%Equivalentemente, \(e\) è un epimorfismo se è cancellabile a destra, ovvero \(f \cmp e = g \cmp e\) implica \(f = g\) ogni qual volta la scrittura ha senso (ovvero quando \(f\) e \(g\) sono frecce parallele e post-componibili con \(e\)).
\end{definition}



\section{Sezioni e retrazioni}

\begin{definition}[Sezione]
	Una \emph{sezione} di un morfismo \(f \colon A \to B\) in una categoria \(\ctC\) è un'inversa destra di \(f\), ovvero un morfismo \(s \colon B \to A\) tale che \(f \cmp s = \id_{B}\).
\end{definition}

\begin{definition}[Retrazione]
	Una \emph{retrazione} di un morfismo \(f \colon A \to B\) in una categoria \(\ctC\) è un'inversa sinistra di \(f\), ovvero un morfismo \(r \colon B \to A\) tale che \(r \cmp f = \id_{A}\).
\end{definition}